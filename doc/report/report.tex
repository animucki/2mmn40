\documentclass[11pt,a4paper]{article}
\usepackage[latin1]{inputenc}
\usepackage{amsmath}
\usepackage{amsfonts}
\usepackage{amssymb}
\usepackage{graphicx}

%code listings setting 
\usepackage{listings}
\lstset{
	breaklines=true
}

\author{Bartosz Animucki {\tt (1226153)}, \\ Roman Kazus {\tt (1239966)}}
\title{2MMN40 Report: Modeling aqueous ethanol}

\begin{document}
	\maketitle
	
	\section{Introduction}
	Ever since ethyl alcohol was successfully distilled in high concentrations for the first time during the Islamic golden age \cite{first_distil}, (al)chemists and enthusiasts alike have been fascinated by its wide uses such as a medical and industrial solvent, as a substrate in organic chemistry, as a psychoactive substance, and many more. 
	
	Nonetheless, perhaps the most important current research goal that involves the study of the properties of ethanol, especially in water mixtures, is biofuels \cite{quovadis} \cite{fuelcell}. Being derived from biomass sources, bio-ethanol contains water, no matter how efficiently it is distilled. This is because at about 95\% ethanol, the mixture is azeotropic -- meaning that neither ethanol nor water evaporate more when the mixture is heated. Further study of this phenomenon, as in \cite{azeobreak}, would be of immense value to the feasibility of biofuel production. 
	
	The exact chemical properties of this type of water-ethanol mixtures, such as volume contraction, have also been studied extensively. One prominent theory of the origin of aqueous ethanol volume contraction, put forth by Frank and Evans \cite{frankevans}, hypothesized that water molecules form so-called ``icebergs'' around hydrophobic solute molecules. Under this theory, when ethanol molecules are introduced to water, the water molecules around the hydrophobic  end  of  the  ethanol  undergo  a  structural rearrangement in such a way that strong water-water hydrogen bonds are formed. This model was repeatedly tested and verified later on in prominent experimental studies such as \cite{parke}. Similar results were found in \cite{guoetal}.
	
	In this report, we propose a basic computational model for predicting thermodynamic properties of water-ethanol mixtures using the principles of Molecular Modeling.
	
	\section{Theory}
	
	\section{Simulation}
	
	\section{Results and discussion}
	Verify simulation against \cite{guoetal}
	
	\section{Conclusion}
	
	
	
	\bibliography{bibliography}
	\bibliographystyle{alpha}
	
	\appendix
	\clearpage
	\section{Code}
	\lstinputlisting[language=Python]{inputCode.py}
	
\end{document}